\documentclass[fleqn, 12pt,letterpaper]{article}
\title{Potential Impacts of Interest in Big Data on Africa's Mobile Phone Revolution}
\author{Reflection 4 \\Kai Nicoll-Griffith[40012407]}

\renewcommand{\thesubsection}{\thesection.\alph{subsection}}
\linespread{2}


\begin{document}
	\maketitle
	
	
\hspace*{10mm} Just as a Westerner loves their social media platforms from Facebook to Twitter, in Africa there is a similar mainstream love of mobile phones and revolutionizing them for all purposes. With this rapid growth comes much wealth and innovation for the stakeholders in the technology, however with this growth comes concerns of the people with interests in political power and their desire to upset the established governments or maintain control over it. Just like in social media our Western nations must concern ourselves with the impact of intelligent propaganda on the Internet, the Big Data fascination could have serious reprocutions on the stability of Africa as malignant political agents use emerging cellphone technology as a vector into the mind of their citizens. These malicious political agents will learn about their neglected citizen's opinions and craft powerful propaganda messages to manipulate them into either upsetting the current order no matter how much it benefits them or convince them that it is in their best interest to ignore human rights. \\
	\\
\hspace*{10mm} 
In \textit{The Mobile Phone 'Revolution' In Africa: Rhetoric or Reality?} his final section of the article talks of \textit{the limits of mobile phones and limited opportunities} which points out two main areas of concern. The effects of a state monopoly on cellphone providers and the use of mobile phones as a weapon for war on destabilizing nations. It's important to question if the mobile phones of Africa could be considered as an agent of big data. In the context of Zwitter's criterion points 1 and 2 of there being a lot of data available and that it is organic data from the world itself is self evident in that individuals call it a revolution in mobile phones. A revolution implies that it is being used in all facets of life, therefore data from mobile phones would be in large quantity and organic. His point 3 that big data is potentially global must be downsized to include continental as the mobile phone revolution is limited in scope and that the data could imply causation is a matter of how it could be hypothetically implemented.\\
\hspace*{10mm} Given that we have established that the mobile technology could be used for big data, the question is who would benefit from it. In Etzo and Collender's briefing they point out as the last argument in the piece that mobile phones were used as a weapon of war to make individuals question the 2007 Kenyan election. In our society Zwitter pays reference to RIOT as a potential concern in Western freedom and privacy. Both politically violent radicals who wish to install their own regime and western politicians eager to maintain their popularity with voters both resort to using information technology to better their end goals. Both would use Big Data if it means that they can advance there positions. Some African countries given their monopoly on the cell phone infastructure would undoubtably use it if it could benefit them\\
\hspace*{10mm} So the question then becomes, what could big data's impact be on African governments and radicals seeking to advance their positions. For an incumbant government they may look at studies in propensity to perform ideological crackdowns and expose people with the possibility of holding certain points of view. For a radical group they may use this data for the purpose of manipulation of people into making uninformed decisions about the current government and entice people to rebel without cause. In both cases, the missuse of big data could benefit African country's taking advantage of mobile phone data.\\
\hspace*{10mm} In conclusion, there are certain concerns to be looked at when observing the mobile phone revolution of Africa. If big data is to get involved there will be a chain of events leading to inevitable political violence and ideological suppression. It is in the authors opinion that while the growth in mobile phones has the potential to finally unify groups of Africans who may hate one another simply on culture, to put aside their differences for the greater good, but like all technology, it can be used either way for the suppression of man or it's advancement. It all depends on those who we support and those we publicly disavow. 

	 \pagebreak
	
	\bibliographystyle{plain}
	\bibliography{}
	
	
	Zwitter, A. (2014). Big Data ethics. \textit{Big Data \& Society, 1}(2),  205395171455925.\\ \hspace*{10mm} doi:10.1177/2053951714559253 \\
	Etzo, S., \& Collender, G. (2010). The mobile phone revolution in Africa: \\ \hspace*{10mm}Rhetoric or reality? \textit{ African Affairs, 109}(437), 659-668. doi:10.1093/afraf/adq045
	




	
\end{document}

